\chapter{Simulations}
\label{ch:simulations}

Numerical simulations have emerged as a new tool to do science, alongside theory and experiments.
In his review, \cite{Heng2014} poses himself the question "Is numerical mimicry a third way of establishing truth?``. Unlike other, laboratory-based disciplines, astronomers may not exert full control over their experiments — one simply cannot rearrange objects in the sky.




\section{Smoothed Particles Hydrodynamics}
Smoothed Particles Hydrodynamics (SPH) is a finite volume, particle based numerical method to solve the Navier-Stokes equation of motions for a fluid.
It is a Lagrangian method because the elements carrying information about the fluid move along the fluid itself.

Other methods exists to discretize the fluid domain, like Adaptive Mesh Refinement (AMR).
On the other hand SPH is meshfree.

Recently, more flexible methods (which come with their own difficulties) have emerged \citep{Springel2010, Shadowfax, Arepo}.
Several research groups are pushing forward different techniques, sometimes following their own tradition and claiming . 
Usually simple test problems (Sod tube (1D,2D), Sedov blast, Kelvin-Helmholtz instabilities, Noh test, Gresho vortex \citet{Gresho1990}), %TODO fix test
which can be solved analytically, can serve as a benchmark for the accuracy of the numerical solution. This is for example done with an L2-norm \citep{Borrow}.
Numerical solutions are always a tradeoff between the accuracy and the practicality.
It is interesting to see how a certain gain in accuracy is translated in an increase in \emph{time-to-solution}. the But at the end of the day if we pay attention to the the trend is 

SPH has been originally developed as a \emph{probabilistic} particle method for simulating astrophysical problems \citep{Lucy1977, Gingold1977}.

In his original treatment \citet{Gingold1977} the probabilities of were defined as 
\begin{equation}
 \mathcal{E}[\rho] % TODO expected values
\end{equation}





\section{Density Estimation from particles ensemble}
A possible choice for the kernel is a gaussian.
% TODO kernel
It's much more practical to use a finite support approximation of a gaussian.
The most used for SPH are the Schoenberg \cite{Schoenberg1988} B-spline function, generated as the Fourier transform %TODO of what?
\citep{Price2012}:
\begin{equation}
M_n(x, h) = \frac{1}{2\pi}\int_{-\infty}^{\infty}{\left(\frac{\sin(kh/2)}{kh/2}\right)\cos(kh)\d k}
\end{equation}

It is convenient to write the kernel as a function of $q$, the distance normalized by the smoothing length: $q=x/h$.
The most widely used is the cubic spline kernel $M_4$:
\begin{equation}
M_4(q) = \frac{1}{\pi h^3} \left\{ \begin{array}{ll}
\frac{1}{4}(2-q)^3 - (1 - q)^{3}, & 0 \le q < 1; \\
\frac{1}{4}(2-q)^3, & 1 \le q < 2; \\
0. & q \ge 2, \end{array} \right. \label{eq:cubicspline}
\end{equation}

\paragraph{Role of the normalization term}
% TODO probabilistic treatment if it fits
It is somehow striking that the normalization term is not the default in current visualization routines, the only reason being spurious effects when dealing with free surfaces \citep{Price2007}.

\begin{equation}
 A(\vect r) = \dfrac{\sum_j \frac{m_j}{\rho_j} A_j W(|\vect r - \vect r_j|,h)}{\sum_j \frac{m_j}{\rho_j} W(|\vect r - \vect r_j|,h)}
\end{equation}

