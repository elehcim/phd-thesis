% !TEX root = thesis.tex

\chapter{Conclusions and future work}
\label{ch:conclusions}
We sum up here the main conclusions of the research carried out in this PhD effort, and we briefly give an overview of the ongoing research projects.\\[5ex]


\section{Conclusions}

\begin{itemize}
  \item we applied for the first time the Moving Box technique (see Section~\ref{sec:MovingBox}) to a galaxy-cluster setup, being able to simulate gas-rich realistic late-type galaxies falling into a Fornax-like cluster (Chapter~\ref{ch:simulations});
  \item analyzing the results of the simulations we found that the jellyfish phenomenon is a relatively short transitory phase of the galaxy along its orbit, and its likely a precursor of the transformation of a dwarf galaxy into an UDG (Chapter~\ref{ch:sim_results});
  \item we were able to propose an estimate of the the most likely position of an NGC~1427A-like galaxy in front of the cluster center, and its most likely flight direction (Chapter~\ref{ch:ngc1427a});
  \item we contributed to the development of novel low-dimensional manifold extraction technique from cloud points and applied the method to the analysis of the tails of a simulated jellyfish galaxy (Chapter~\ref{ch:manifolds}).
\end{itemize}

\section{Ongoing research efforts}
\subsection{MUSE mock cube}
Integral-Field Units (IFU) are powerful instruments which couple the discovery potential of imaging devices to the measuring capabilities of spectrographs.
MUSE (Multi Unit Spectroscopic Explorer is an integral-field spectrograph operating in the visible wavelength range, currently mounted on the Very Large Telescope (VLT) of the European Southern Observatory (ESO) \citep{Bacon2010, MUSEWebpage}.
As a way to compare simulations with observations and check the tools currently employed in the analysis of IFUs data cubes, we created a tool to produce MUSE-like data cubes from the MoRIA suite of simulations.
Code is freely available from \url{https://github.com/elehcim/simifucube}.
This work has been done in collaboration with the Heidelberg node of the SUNDIAL network.

We start by taking the star particles of a simulated snapshot.
Similarly to \citet{Ibarra-Medel2019}, depending on the age $t_i$ and metallicity $\feh_i$ of each star particle $i$, we assign to it a spectrum from the EMILES library \citep{Vazdekis2010}, with intensity $S_i$ for each \emph{emitted} wavelength $\lambda_e$: in practice the spectrum can be written as $S_i(\lambda_e;\feh_i, t_i)$.
%Where $S_L$ is the “perfect” noiseless spectrum built from the spectral library.
We use the library of spectra modeled with a Chabrier IMF with slope $1.3$ and BaSTI isochrones \citep[][Bag of Stellar Tracks and Isochrones]{Pietrinferni2013}.
Other spectral libraries are easily configurable if needed.
Depending on the redshift $z$ and on the current line-of-sight velocity $V_{\mathrm{los},i}$ of the particle, the spectrum should be Doppler shifted, obtaining the \emph{observed} flux $F_i(\lambda)$:

%\begin{equation}
%  F(\lambda) = \sum_{i=0}^{N} \frac{S_i(\lambda\,(1 + z + V_{\mathrm{los},i}/c),\feh_i, t_i) m_i }{4\pi d},
%  \label{eq:shifted_flux}
%\end{equation}
\begin{equation}
  F_i(\lambda) = \frac{S_i\left(\lambda_e\,(1 + z + V_{\mathrm{los},i}/c);\feh_i, t_i \right) m_i }{4\pi d},
  \label{eq:shifted_flux}
\end{equation}
with $d = 20$~Mpc is the distance from the galaxy, equivalent to the Fornax cluster assumed distance and $m_i$ the star particle mass.

After binning the wavelength dimension in $K=10000$ channels $j$, for each frequency $\lambda_j$, at each spatial position $(x,y)$ in the plane of the sky we compute the projected SPH interpolation along the $z-$axis of the observed spectra, producing a SPH map using eq. \eqref{eq:projected_kernel}:
\begin{equation}
  F(x,y, \lambda_j) = \sum_{i=0}^{N} \tilde{W} \left(\sqrt{(x-x_i)^2+(y-y_i)^2}, h_i \right) F_i(\lambda_j),
\end{equation}
where N is the number of star particles in the snapshot and $(x_i, y_i)$ their projected position.
Our tests have been carried out with a spatial grid of $80\times 80$ spaxels.

Given the amount of frequency channels in the spectral library, we had to carefully optimize the routine to perform SPH projected interpolation.
In fact, the code presented in the repository \citet{simifucube} uses some \textsc{pynbody} \citep{Pontzen2013} routines modified and optimized for spectral data, and heavily uses the \textsc{SpectralCube} tool \citep{SpectralCube}.

We then convolve each spaxel with the MUSE line spread function (LSF) \citep[$F_{\mathrm{udf}10}$, eq. (8) in][]{Bacon2017}.
We finally add to the the datacube value a zero centered normal distributed contamination $\mathcal{N}$ with dispersion $\sigma$:
%\begin{equation}
%  S = S_L + \mathcal{N}(0, S_L\cdot \sigma )
%\end{equation}
\begin{equation}
  F(x,y, \lambda_j) = F(x,y, \lambda_j) + \mathcal{N}(0, F(x,y, \lambda_j)\cdot \sigma )
\end{equation}

We defined the contamination dispersion as the dispersion of the residuals of the fit for a typical galaxy (from \citet{Bidaran2020} dataset we take the VCC~1836 bin \#70 with initial $S/N=3$ and target $S/N=40$).
This resulted in a dispersion of $\sigma = 0.073$.
Eventually, we create the \verb|STAT HDU| in the datacube using \verb|DER_SNR| algorithm \citep{Stoehr2008}. The noise model chosen in this stage is wavelength independent.

%\subsubsection{Preliminary results}
%With the collaboration of the Heidelberg node.


%\section{Future perspectives}
%\label{sec:futurework}
